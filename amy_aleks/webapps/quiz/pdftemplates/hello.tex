\documentclass[12pt, a4paper]{article}
\usepackage[x11names]{xcolor}
\usepackage[T1]{fontenc}
\usepackage{xltxtra}
%\usepackage{xunicode}

%\XeTeXinputencoding "cp936"

%\usepackage[top=1.2in,bottom=1.2in,left=1.2in,right=1in]{geometry} % 页边距
\defaultfontfeatures{Mapping=tex-text}%连字号

\usepackage[boldfont,slantfont,CJKchecksingle]{xeCJK} % 允许斜体和粗体
\xeCJKsetup{PunctStyle=hangmobanjiao}
\setlength{\parindent}{0cm}                 % Default is 15pt.
\linespread{1.2}                       % 行间距
\setlength{\parskip}{\baselineskip}    % 段间距

\XeTeXlinebreaklocale "zh"
\XeTeXlinebreakskip = 0pt plus 1pt minus 0.1pt

\usepackage[unicode=true,colorlinks,linkcolor=blue]{hyperref} % 超链接
\setCJKmainfont[BoldFont=SimHei, ItalicFont=KaiTi]{SimSun}    %配置中文字体
%\setmainfont{Times New Roman}                % 英文衬线字体
%\setmonofont{Courier New}                    % 英文等宽字体
%\setsansfont{Droid Sans}                     % 英文无衬线字体


\usepackage{graphicx}                   % 嵌入png图像
\usepackage{longtable,tabu,booktabs}
\usepackage{pdflscape}

\usepackage{tocloft}                        % 目录
\renewcommand\contentsname{目录}
\renewcommand{\today}{\number\year 年 \number\month 月 \number\day 日}

\begin{document}

\title{\textbf{测试文档}}
\author{作者名}
\maketitle
\date


%\section{莫迪访上海}
%5月14日至19日,印度总理莫迪将对中国、蒙古和韩国进行访问。
%这是莫迪上任后首次作为印度总理出访中国,此前他曾多次以古吉拉特邦首席部长的身份来华。

%\section{克里提出对中国南海填海关注 \ 王毅:不要有误判}
%克里提出对中国在南海填海的关注,指希望各方采取降低紧张局势的做法。
%而针对美国在南海问题上立场的变化,王毅重申中国维护主权领土是正当行为,
%并指以和平外交手段在直接当事国之间寻求妥善解决的立场不会改变,
%王毅说中美都希望维护南海和平稳定,中国与东盟和美国在这些方面的对话都可以继续,
%重要的是不要有误判。


\section{ {{ q.body }} }

{{ option.order }}
{{ option.body }}





\end{document}

